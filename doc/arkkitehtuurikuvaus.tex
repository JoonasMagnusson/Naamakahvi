\documentclass[11pt]{article}
\usepackage[utf8]{inputenc}
\usepackage[T1]{fontenc}
\usepackage{graphicx}
\usepackage{longtable}
\usepackage{float}
\usepackage{wrapfig}
\usepackage{soul}
\usepackage{amssymb}
% \usepackage{hyperref}


\title{Architectural design document\\
  Software Engineering Project Course\\
  Naamakahvi\\
  University of Helsinki}

\author{Antti Hietasaari
  \and Joonas Magnusson
  \and Irina Mäkipaja
  \and Samir Puuska
  \and Janne Ronkonen
  \and Eeva Terkki
  \and Ossi Väre}
\date{\today}

\begin{document}

\maketitle

% \setcounter{tocdepth}{3}
\tableofcontents
% \vspace*{1cm}



\section{Introduction: Architectural Overview}
% \begin{itemize}
% \item High-level description of overall architecture
% \item Main functionalities described in natural language
% \end{itemize}

The staff of the Department of Computer Science has two coffee syndicates, both of which provide an espresso machine. The other syndicate also has a drip coffee maker. The members pay for their coffee by bringing products to be used in the machine, e.g. espresso beans or filters.

\section{System Purpose: Requirements}


\subsection{Functional Requirements}
   % --> Bullet point list
   % --> e.g. must be able to bill customer credit cards, must be able to
   % insert a code snippet to customers site. etc.

\begin{itemize}
\item{The user can register on the system}
\item{The user can be authenticated using face recognition}
\item{The user can be authenticated using their username}  
\item{The user can buy one or more coffees}
\item{The user can bring products}
\item{The user can check their balance}
\item{The user can add new images after registration}
\end{itemize}  

\subsection{Non-functional Requirements}
   % --> Bullet point list
   % --> e.g. expected unit test ratio, expected test coverage, expected
   % performance

\begin{itemize}
\item{The application can be used on an Android tablet or a desktop equipped with touch screen}
\item{The software is simple to use and the core functionalities do not require many clicks}
\item{The user does not have to wait for very long}  
\end{itemize}

\section{Structure}

\subsection{Overview: overall structure}
% --> Diagram of all components and their collaboration with each other
\includegraphics[scale=0.5]{overall-structure.png}


\subsection{Components}
   % for each component:
   %   - description of the component
   %   - components responsibilities
   %   -interfaces that the component offers
   %   - constraints
   %   - collaboratoring components
   %                for each collaborating component:
   %                  -  description of the collaboration
   %                  - interfaces used
   %                  - constraints

   % --> Just short listing of every component what should say,
   % e.g.
   % KujeProcessor is responsible for XX and YY, 
   % interfaces: removes processed data from kuje-table, inserts data for WebApp,
   % collaborates: with KujeServer and WebApp, but KujeServer does not
   % collaborate (directly) with WebApp.
   % collaborates:
   % KujeServer: takes kuje records from KujeServer database and afterwards
   % removes them
   % WebApp: updates the reports which are represented by WebApp
   % for each component interface:
   %   - description of the interface
   %   -  operations
   %   - constraints on the order of operation

\subsubsection{Client/frontend}
The client is implemented as a Java library and its responsibility is to offer an interface
for UI-components, abstracting away the communication protocols used between the client and server.

The client communicates with the server via HTTP, sending GET and POST requests, and receiving JSON-objects from the server.


\subsubsection{Backend/server}
% redox ja mahnu vois varmaan kirjoittaa tähän jotain tarkempaa, ja
% ehkä jakaa tämän kohdan useampaan kohtaan jos siltä tuntuu.
The server updates the database and gets data from it on client requests and tells the client whether or not the requested operation was successful.

\subsubsection{Android-UI}
% android-ui ihmiset voisivat kirjoittaa tästä
\subsubsection{Swing-UI}
% swing-ui ihmiset voisivat kirjoittaa tästä
     

\section{Dynamic Behavior}
\subsection{Scenarios}
% for each scenario:
% - type: system operation or use case
% - description of the scenario
% - how scenario interacts with components

\subsubsection{Registration}
Type: Use case\\
Description:\\
Component interaction:\\
   
\subsubsection{Authentication using face recognition}
Type: Use case\\
Description:\\
Component interaction:\\
   
\subsubsection{Authentication using username}
Type: Use case\\
Description:\\
Component interaction:\\

\subsubsection{Buying coffee}
Type: Use case\\
Description:\\
Component interaction:\\

\subsubsection{Bringing a product (paying)}
Type: Use case\\
Description:\\
Component interaction:\\

\subsubsection{Adding new photographs}
Type: Use case\\
Description:\\
Component interaction:\\


\section{Other Views}


\subsection{Process: in a running system, how components are divided as processes}
% --> e.g. There are can be multiple instances of KujeProcessor, but can
% there be multiple instances of KujeServer?
% --> Has multiple instances of KujeProcessor ever been tested? Has
% multiple instances of KujeServer ever been tested?

\subsection{Development: Where in code the codebase components are implemented}
\subsection{Physical: How components are separated in the hardware level}
% --> e.g. Division between the frontend and the backend; list also all physical
% dependencies between components, e.g. ``KujeProcessor and WebApp must stay
% on the same physical server because ...

\subsection{Deployment: How components are deployed and possible constraints}

\end{document}